
\subsection{Research Question 1}
The most important achievement regarding the first research question is that we successfully confirmed that we still keep the economical interpretability of the RF models while using its advantage to create a better predictive framework. Given the novel nature of the approach in this study, it is important to acknowledge that there are not a lot of literature to compare the interpretability aspect. This is especially true for the interpretability of the RF. Most of the contemporary literature either enhance the tradition OLS asset pricing models with liquidity risks and sentiment, often separately or use various ML techniques to test the forecasting robustness of the asset pricing models \cite{gu_2020,acharya_2005,delong_2019}. Even the paper that the approach of this thesis is based on, \citeA{simonian_2019} do not go into quite depth on the effectiveness of the pseudo beta approach, or statistical comparison between the OLS betas and RF pseudo betas. That said, this thesis provides a novel approach by creating a large set of models, then applying the DM test to compare the forecasting performance of the multitude of models to statistically confirm that the RF models do add predictive power to OLS, and that the enhanced RF models perform better forecast than the baseline. The performance comparison is then combined with fit and error statistics, which with the backing of the DM test is much more rigorous than just using on the training metrics alone. With the rejected null hypotheses of the DM tests on OLS vs RF we can say that RF does improve predictive power and fit, and we can say that the interpretability of the RF could be retained and be analogous to the common coefficients financial experts use. However, because the DM test fails to reject the Hypothesis \ref{hyp:dm3}'s null hypothesis of equal predictive accuracy between the RF base and RF enhanced models in certain sectors, this thesis only partially confirms the incremental value of the liquidity and sentiment features. Explaining why the forecasts of RF base and RF enhanced do not differ statistically in these sectors would require a detailed, sector-by-sector analysis of their specific conditions and percentile dynamics, which is beyond the scope of this study. Therefore, this thesis suggests the following paths for future research. 

The liquidity risk and sentiment factors added are well-established in the literature and have been proven to work, in both traditional Fama French and in other non-parametric settings. However, with the robustness of RF in our case, it would be entirely possible to produce other factors to add into this set of factors. Given that future research should extend the training window of the models to more recent years, it would be a must to include COVID-19 macroeconomic shocks to the model. It would also be interesting to see how these COVID proxies interact with the liquidity and sentiment factors, and whether they would be able to provide additional predictive power to the model. Other current thematic factors that could be important would be sustainability risk for firms that have poor environmental, social and governance profiles.


\subsection{Research Question 2}

Due to the sector rotation implementation failing to reject Hypothesis \ref{hyp:psr}, there is no statistical evidence that the enhanced RF models improve sector rotation strategies. Although all RF variants delivered impressive trading performance of the S\&P500 during the 2018 hold-out period, the excess returns are not sufficient to justify real-world implementation, as they do not adequately compensate investors for the risks assumed. Given the lack of statistical support for enhanced RF-based sector rotation, it is therefore useful to explore alternative rule-extraction frameworks and an extended training window.


The ARL relies on manually selected support and confidence thresholds, which may render the extracted “if-then” sector-rotation rules sensitive to arbitrary parameter choices. Recent advances, such as the RuleFit algorithm, which generate sparse, additive rule ensembles directly from the data without the need for external rule mining and threshold tuning. By integrating rule extraction into the learner itself, RuleFit produces highly concise rules that can be directly interpreted with the variables in the model itself, without backtracking to RF to find exactly which condition leads to a certain excess return(e.g.\ \emph{if} market return $>$ $\alpha$ \emph{and} turnover $\le \beta$ \emph{and} sentiment $>$ $\gamma$, \emph{then} excess return $=10\%$) while avoiding the two step RF+ARL pipeline. However, due to the limited computational resources and the recorded inability to produce robust forecast, it has been determined that this rule extraction method is not feasible for the research questions of this thesis \cite{molnar_2019}. 

Despite implementing an expanding-window training method, our backtest remains confined to a single year of walk-forward evaluation due to computational constraints. Ideally, we would train on an initial lookback of $X$ trading days, test on the subsequent day, then extend the training window and repeat this process continuously over a 20-year period. While this paper reports out of bag performance for a one-year training frame across two decades, the absence of a full dataset rolling back test in our study limits the robustness of our conclusions. Hence, we were unable to statistically confirm that the sector rotation signals generated with ARL and RF outperform the S\&P500. Therefore, based on the current findings, the second research question cannot be definitively answered. Future research could address this gap by incorporating the infinitesimal jackknife (IJ) variance-estimation framework of \citeA{wager_2014}. The IJ observes how your FI (there by pseudo-beta) estimate would shift if bootsrap is applied to the training data using established trees and uses those shifts to work out how much the estimate would naturally vary from one sample to the next.  Once these standard errors are available, a simple ratio of the estimate to its estimated standard error yields a $t$-statistic, enabling formal confidence intervals and hypothesis tests on both the magnitude and the sign of the pseudo-beta coefficients. A full IJ implementation, however, would require thousands of additional bootstrap replicates and substantial computational power, placing it beyond the scope of the present study. Building on the IJ variance framework, future work should execute a true expanding-window back-test at the daily frequency, starting with the first trading days in 1991 as \citeA{simonian_2019} and rolling forward one day at a time until 2025. Incorporating the most recent \texttt{CRSP} and \texttt{Compustat} releases together with pandemic era controls, such as a COVID-19 indicator/proxies, mobility indices, and fiscal-stimulus shock proxies, would capture the extraordinary macro-financial regime of 2020-2021 without compromising comparability across the full horizon. This improvement would deliver the sector rotation signals required to determine, with adequate statistical power, whether machine learning supported sector rotation strategies could outperform the S\&P500 and would greatly improve returns for investors.