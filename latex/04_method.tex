%%%OUTLINE:%%%%

- liquidity risk factors construction
- sentiment construction
- fama and carhart models/factors 
- random forest
- Rule based model
- sector rotation strategy: Association rule learning

This chapter outlines the methodological approach adopted to measure liquidity risk and investors' sentiment, along with their rationale. These factors are integrated into an enhanced asset pricing framework based on the FF5 and C4F. Subsequently, the implementation of the Random Forest (RF) algorithm is detailed. Next, the rule-based model will be extracted from the RF for comparison and application to the Association Rule Learning Sector Rotation Strategy.

\textbf{Liquidity Risk factor}
%%%%% ACTUAL%%%%%%%%%%%
The construction of liquidity factors closely align with the methodology outlined by \citeA{gu_2020}. The authors include multiple liquidity related variables, including turnover and turnover volatility (\texttt{turn}, \texttt{SDturn}), log market equity (\texttt{mvel1}), dollar volume (\texttt{dolvol}), Amihud illiquidity (\texttt{ILLQ}), number of zero trading days (\texttt{zerotrade}), and bid-ask spread (\texttt{baspread}). \citeA{amihud_2002} defined their illiquidity measure \texttt{ILLQ} as the "average ratio of the daily absolute return to the(dollar)trading volume on that day. This index captures waves of excessive optimism or pessimism in the market.  The calculation for Amihud illiquidity is as follows:

\begin{equation}
    \label{eq:amihud}
    \text{ILLIQ}_{i,y} = \frac{1}{D_i^y} \sum_{d=1}^{D_i^y} \frac{|R_{i,d}|}{VOL_{i,d}}
\end{equation}
    
with $R_{i,d}$ being the return on stock $i$ on day $d$. $VOL_{i,d}$ being trading volume (in dollars) for stock $i$ on day $d$ and $D_i^{y}$ being the number of days for which data is available for stock $i$ in year $y$. \citeA{amihud_2002} suggested that this measure be multipled with the ratio of $10^6$, as the ratio would be tiny. Each stock-level liquidity characteristic is cross-sectionally ranked period-by-period, then mapped into a standardized interval ranging from -1 to 1 \cite{gu_2020}. 

%(is batch normalization needed?)

%Add a part that says unlike the literature on this topic, no portfolio sorting is needed.

\textbf{Sentiment}
%% OUTLINE: wurgler -> antoniu -> XU enhanced
\citeA{wurgler_2006} constructed a sentiment index by capturing the principal components within six proxies: closed-end fund discount (\texttt{CEFD}), NYSE share turnover (\texttt{TURN}), the number and average first-day returns of IPOs (\texttt{NIPO, RIPO} respectively), equity share in new issues (\texttt{S}) and dividend premium (\texttt{P}). \footnote{However, by March 2025, the NYSE share turnover proxy has been dropped in the database due to "the explosion of institutional high-frequency trading and the migration of trading to a variety of venues"}%can definitely add more after the proposal here, from the paper

The sentiment is calculated as follows:
\begin{equation}
    \label{eq:sentiment}
    \begin{split}
    \text{SENTIMENT}_t = -0.241\,\text{CEFD}_t + 0.242\,\text{TURN}_{t-1} \\ + 0.253\,\text{NIPO}_t 
    + 0.257\,\text{RIPO}_{t-1} \\ + 0.112\,S_t - 0.283\,P^{D-ND}_{t-1} , 
    \end{split}
\end{equation}
where each proxy has been standardized. $\text{D}-\text{ND}$ is the difference between dividend payers and non-dividend payers. Since both market sentiment and the business cycle can drive common variation in financial data, the Principal Component Analysis(PCA) will treat both as sources of variance without distinguishing whether that variance comes from changes in investor sentiment or broader macroeconomic factors. A second orthogonalized index is formed by regressing each of the proxy with independent variables that explains the business cycle. The residuals of this regression will be the "pure" sentiment, which is the variation that the macroeconomic factors fail to explain. The following is the orthogonalized index, with $\perp$ labelling the removal of business cycle. 
\begin{equation} %BEQ
    \label{eq:sentiment_orth}
    \begin{split}
    \text{SENTIMENT}^{\perp}_t = &-0.198\,\text{CEFD}^{\perp}_t + 0.225\,\text{TURN}^{\perp}_{t-1} \\
    &+ 0.234\,\text{NIPO}^{\perp}_t + 0.263\,\text{RIPO}^{\perp}_{t-1} \\
    &+ 0.211\,S^{\perp}_t - 0.243\,\text{P}^{\text{D} - \text{ND},\perp}_{t-1}
    \end{split}
\end{equation}



%%%%%%%%%OUTLINE%%%%%%%%%%%
This index captures waves of excessive optimism or pessimism in the market. aker and Wurglers key insight was that sentiment affects certain stocks more than others: “when beginning-of-period proxies for sentiment are low, subsequent returns are relatively high” for stocks that are speculative and hard to arbitrage (small, young, high-volatility, unprofitable, non-dividend-paying, or distressed stocks)Conversely, when sentiment is high, those same categories of stocks earn low future returns . during euphoric periods, speculative stocks become overpriced (and later underperform), whereas after pessimistic periods they are underpriced (and subsequently outperform). This pattern suggests that sentiment-driven mispricing occurs and then corrects, especially for stocks that are difficult for arbitrageurs to short or value objectively.Classic finance theory would predict no role for sentiment (prices = fundamental value), but Baker and Wurgler provide evidence that broad waves of optimism/pessimism can create return predictability across stock types, beyond what risk factors explain.


\subsection{Factor models}
%%%%%%%%OUTLINE%%%%%%%%%%
- Fama French 3
- Fama Carhart 4
- Fama French 5

- Factor models are generally linear. Which kinds of regression are they usally used with (fama macbeth)? 
- Since there are debate for which factor model is best, we want to test FF3, C4F and FF5

Time series regression on six asset pricing models.

Seven factor model:

\[
R_{it} - R_{Ft} = \alpha_i + \beta_1 \text{MktRF}_t + \beta_2 \text{SMB}_t + \beta_3 \text{HML}_t + \beta_4 \text{RMW}_t + \beta_5 \text{CMA}_t + \beta_6 \text{IML}_t + \beta_7 \text{LR}_t, \quad (6)
\]

I am going to run also the sentiment index on this too, so 8 factors

" Linear
models are preferred by practitioners because they gen-
erally present readily understandable and interpretable
analysis. In contrast, machine learning approaches,
although useful in uncovering the nonlinear behavior
of and interaction relationships among variables, are
often articulated in a way that makes their output unin-
tuitive, and hence unattractive, to many investment pro-
fessionals. "

Also, you don't even need to do the portfolio sortings like in thesis

Instead of just looking at the average effect of factors, analyzing different percentiles helps us understand how factors behave across low, median, and high return environments.

We are going to do this with the portfolios constructed before, so no need for percentiles


Random Forest Importance compares how important each feature is comparing to others in the same set, it is very useful to offer portfolio guidance such as in  return based style analyis (RBSA). he RF model, however,
possesses an advantage over a standard RBSA in gener-
ally providing a much better fit,

"Because the RF model captures hierarchical
(non-geometric) relationships between factors, it cannot
be understood as a direct analog of an OLS regression
or PCA because it does not convey the individual direc-
tional relationships between factors and assets."

We have to attempt to "beta-tize" the random forest. Beta is basically the elasticity of one variable to another. What we can do is to divide the predicted target variable return by each pre-
dictor return to gain a raw elasticity value for each factor.

\subsection{Liquidity proxy}

\textbf{Amihoud illiquidity ratio and Liquidity factor portfolio}
$$\text{Illiquidity} = \text{Average} \left( \frac{|R_{iyd}|}{\text{VOLUME}_{iyd}} \right)
$$

%%%%%%%%%%ACTUAL%%%%%%%%
One widely used metric is the Amihud illiquidity measure, which measures the price impact of trading. It could be intuitively understood as how much prices move per unit of volume - higher values mean the stock is harder to trade without moving the price. Amihud also found a significant illiquidity premium, which are stocks with higher Amihud's illiquidity values earn higher average returns, presumably compensating investors for bearing liquidity risk. In his cross-sectional tests, expected stock returns increased with expected illiquidity \cite{amihud_2002}


1.
- Use this for every single stock listed on the SP500 on a day as a proxy for stock illiquidity. Multiply this ratio by 10 to power 6. 
- Calc the moving average monthly (21 trading days). The first trading day of the month will thus capture the average illiquidity ratio of the previous month.

2.
- Fama and French approach: Sort the stock based on market cap into 2 groups: small and big (This is basically the SMB variable)
- Then the stock is INDEPENDENTLY sorted based on their illiquid ratio into 5 groups.Each stock is value weighted, meaning that stocks with a higher marketr capitalization gets a larger weight in the portfolio
- Now we have $2*5=10$ value weight portfolio based on size and illiquidity


3.
Now we need to construct long-short liquidity factor portfolio:

- Long illiquid portfolios
- Sell most liquid portfolios

A liquidity long-short portfolio is a trading strategy that profits from the difference in returns between illiquid and liquid stocks

This can be verified with the liquidity factor return, or just the excess return for holding illiquid stocks

$$
\text{Liquidity Factor Return} = \frac{1}{2} \left( R_{S5} + R_{B5} \right) - \frac{1}{2} \left( R_{S1} + R_{B1} \right)
$$
Where:

- \( R_{S5} \) and \( R_{B5} \) are the returns of the **most illiquid** small and big portfolios.
- \( R_{S1} \) and \( R_{B1} \) are the returns of the **most liquid** small and big portfolios.



This whole thing captures the return premium associated with illiquidity. Stocks that are harder to trade often earns higher returns


In addition to the 10 illiquidity portfolios, lets compute illiquidity portfolios based on size. I first sort the stocks on their firm’s market capitalization in two groups, and afterward sort stocks on their illiquidity measure in their size group. This creates five illiquidity portfolios for small stocks and five illiquidity portfolios for big stocks.



\textbf{Fama French Five Factors models}

The following are its factors:

CAPM
- Market risk Factor from NYSE, NASDAQ and AMEX (Data taken from CRSP)

Before constructing the factors, Fama and French make 
1. six value weighted on size (small or big) and book to market ratio (value, neutral, growth)
2. six value weighted on size and operating profitability (robust, neutral, weak )
3.six value weighted on size and ivestment activity

Fama french 3
- Size factor (SMB)
    - The SMB factor is computed by subtracting the average return of the nine big stock portfolios
    (Big Value, Big Neutral, Big Growth, Big Robust, Big Neutral, Big Weak, Big Conservative,
    Big Neutral, and Big Aggressive) from the average return of the nine small stock portfolios
- Value factor (HML)
    -  spread between the average of the two value portfolios and the average of the two growth portfolios

Fama French 5
- Robust minus weak (RMW)
    - calculated by subtracting the average return of the two
    weak operating profitability portfolios from the average return of the two robust operating
    profitability portfolios.
- Conservative minus aggressive (CMA)
    - the difference between the average return of
    the two conservative investment portfolios and the average return of the two aggressive
    investment portfolios 

- Excess returns:
    - Return -risk free rate (RF from the kenneth data)

\subsection{Sentiment}
Enhanced investor sentiment dataset (Sze Nie Ung)








\textbf{Sector rotation strategy}

Apply RF variant to build a sector rotation strategy using ARL (Association rule learning). We can use this to deduct inference rule from epirical data

- RL is a rolling-window learning approach, meaning it continuously updates its understanding of market conditions.
Instead of using a fixed model trained once on historical data, it adapts dynamically by re-estimating relationships every 18 months.
This helps capture changing market dynamics—for example, a factor that worked last year may not work the same way today

Two signals are used:
- Random forest predicted excess within a portfolio
- Ratio of shorter-term to longer-term realized volatility (24-month vs. 36-month)

iterative add factor de xem la feature importance co doi ko 