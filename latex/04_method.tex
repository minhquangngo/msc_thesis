%%%OUTLINE:%%%%

- liquidity risk factors construction
- sentiment construction
- fama and carhart models/factors 
- random forest
- Rule based model
- sector rotation strategy: Association rule learning

This chapter outlines the methodological approach adopted to measure liquidity risk and investors' sentiment, along with their rationale. These factors are integrated into an enhanced asset pricing framework based on the FF5 and C4F. Subsequently, the implementation of the Random Forest (RF) algorithm is detailed. Next, the rule-based model will be extracted from the RF for comparison and application to the Association Rule Learning Sector Rotation Strategy.

\subsection{Liquidity Risk factor}
%%%%% ACTUAL%%%%%%%%%%%
The construction of liquidity factors closely align with the methodology outlined by \citeA{gu_2020}. The authors include multiple liquidity related variables, including turnover and turnover volatility (\texttt{turn}, \texttt{SDturn}), log market equity (\texttt{mvel1}), dollar volume (\texttt{dolvol}), Amihud illiquidity (\texttt{ILLQ}), number of zero trading days (\texttt{zerotrade}), and bid-ask spread (\texttt{baspread}). \citeA{amihud_2002} defined their illiquidity measure \texttt{ILLQ} as the "average ratio of the daily absolute return to the(dollar)trading volume on that day. This index captures waves of excessive optimism or pessimism in the market.  The calculation for Amihud illiquidity is as follows:

\begin{equation}
    \label{eq:amihud}
    \text{ILLIQ}_{i,y} = \frac{1}{D_i^y} \sum_{d=1}^{D_i^y} \frac{|R_{i,d}|}{VOL_{i,d}}
\end{equation}
    
with $R_{i,d}$ being the return on stock $i$ on day $d$. $VOL_{i,d}$ being trading volume (in dollars) for stock $i$ on day $d$ and $D_i^{y}$ being the number of days for which data is available for stock $i$ in year $y$. \citeA{amihud_2002} suggested that this measure be multipled with the ratio of $10^6$, as the ratio would be tiny. Each stock-level liquidity characteristic is cross-sectionally ranked period-by-period, then mapped into a standardized interval ranging from -1 to 1 \cite{gu_2020}. 

%(is batch normalization needed?)

%Add a part that says unlike the literature on this topic, no portfolio sorting is needed.

\subsection{Sentiment}
%% OUTLINE: wurgler -> antoniu -> XU enhanced
\citeA{wurgler_2006} constructed a sentiment index (BW index hereafter) by capturing the principal components within six proxies: closed-end fund discount (\texttt{CEFD}), NYSE share turnover (\texttt{TURN}), the number and average first-day returns of IPOs (\texttt{NIPO, RIPO} respectively), equity share in new issues (\texttt{S}) and dividend premium (\texttt{P}). \footnote{However, by March 2016, the NYSE share turnover proxy has been dropped in the database due to "the explosion of institutional high-frequency trading and the migration of trading to a variety of venues" \cite{ung_2023}}%can definitely add more after the proposal here, from the paper

The sentiment is calculated as follows:
\begin{equation}
    \label{eq:sentiment}
    \begin{split}
    \text{SENTIMENT}_t = -0.241\,\text{CEFD}_t + 0.242\,\text{TURN}_{t-1} \\ + 0.253\,\text{NIPO}_t 
    + 0.257\,\text{RIPO}_{t-1} \\ + 0.112\,S_t - 0.283\,P^{D-ND}_{t-1} , 
    \end{split}
\end{equation}

where each proxy has been standardized. $\text{D}-\text{ND}$ is the difference between dividend payers and non-dividend payers. Principal Component Analysis(PCA) reveals the hidden common factor from this group of factors by transforming the data into principal components (PCs) that would capture as much variance in the data as possible. PC1 should capture the maximum variation from the sentiment proxies, thereby serving as an aggregate measure of investor sentiment. Since both market sentiment and the business cycle can drive common variation in financial data, the PCA will treat both as sources of variance without distinguishing whether that variance comes from changes in investor sentiment or broader macroeconomic factors. A second orthogonalized index is formed by regressing each of the proxy with independent variables that explains the business cycle. The residuals of this regression will be the "pure" sentiment, which is the variation that the macroeconomic factors fail to explain. The following is the orthogonalized index, with $\perp$ labelling the removal of business cycle. 

\begin{equation} %BEQ
    \label{eq:sentiment_orth}
    \begin{split}
    \text{SENTIMENT}^{\perp}_t = &-0.198\,\text{CEFD}^{\perp}_t + 0.225\,\text{TURN}^{\perp}_{t-1} \\
    &+ 0.234\,\text{NIPO}^{\perp}_t + 0.263\,\text{RIPO}^{\perp}_{t-1} \\
    &+ 0.211\,S^{\perp}_t - 0.243\,\text{P}^{\text{D} - \text{ND},\perp}_{t-1}
    \end{split}
\end{equation}

However, \citeA{ung_2023} pointed out several problems with the BW index. First, though it has robust predictive perfomance on a cross- sectional level, it is weak for aggregate market retuerns in time series regressions - even \citeA{wurgler_2007} pointed this out themselves in the paper. Second, the BW index assumes that the contributions of each sentiment proxy to the aggregate index are fixed over time. Finally, the index has `look ahead bias': PCA uses the whole sample, and forecasts at time $t$ should not rely on data that would only become available after $t$. Therefore, \citeA{ung_2023} constructed an enhanced index to address these problems. The time-varying BW sentiment index ($S^{TV}$ hereafter) is constructed on a three years rolling window basis to use the most up-to-date information at each $t$, and is built upon \cref{eq:sentiment_orth}. This rolling window allows the model to adjust to structural breaks in the market without distorting the sentiment index. Furthermore, adjustments to the sign of the RICO proxy display a negative initial loading. This paper will use \citeA{ung_2023} $S^{TV}$ index to measure investors' sentiment.


\subsection{Factor models}
%%%%%%%%OUTLINE%%%%%%%%%%
- Fama Carhart 4
- Fama French 5

%%%%%%ACTUAL%%%%%%
This thesis will employ two CAPM variants: the C4F of \citeA{cahart_1997} and FF5 of \citeA{ff5_2015}. The following are their respective formulas:

\begin{equation}
    \label{eq:c4f}
    \begin{split}
        R_{i,t} - R_{f,t} &= \alpha_i + \beta_{MKT} (R_{MKT,t} - R_{f,t}) \\
        &\quad + \beta_{SMB} \cdot SMB_t + \beta_{HML} \cdot HML_t \\
        &\quad + \beta_{MOM} \cdot MOM_t + \epsilon_{i,t}
    \end{split}
\end{equation}

\begin{equation}
    \label{eq:ff5}
    \begin{split}
        R_{i,t} - R_{f,t} &= \alpha_i + \beta_{MKT} (R_{MKT,t} - R_{f,t}) \\
        &\quad + \beta_{SMB} \cdot SMB_t + \beta_{HML} \cdot HML_t \\
        &\quad + \beta_{RMW} \cdot RMW_t + \beta_{CMA} \cdot CMA_t + \epsilon_{i,t}
    \end{split}
\end{equation}

where \cref{eq:c4f} is the C4F and \cref{eq:ff5} is the FF5. $R_{i,t}$ is the return on security and $R_{f,t}$ is the risk-free return; which makes $R_{i,t} - R_{f,t}$ is the excess return on a security. $SMB$ (small minus big) is the size factor, capturing the excess returns of small-cap stocks over large-cap stocks, reflecting how higher volatility and growth potential could cause smaller firms to yield higher average returns. $HML$ (high minus low), or the value factor, accounts for the premium earned by stocks with high book-to-market ratios, often associated with underperforming firms that carry higher risk and, consequently, higher expected returns \cite{ff3_1993}. $MOM$ is the one year momentum factor \footnote{In literature, researchers can refer to this factor as PR1YR, UMD (ups minus downs), or WML (winners minus losers.)} \cite{cahart_1997}. In response to \citeA{titman_2004} and \citeA{novymarx_2013} criticisms, two additional factors were added in to the FF3 model \cite{ff5_2015}. $RMW$ (robust minus weak) is the profitability factor, which derives from the empirical observation that firms with higher operating profitability tend to earn higher average returns. $CMA$ (conservative minus aggressive) is the investment factor, which captures the tendency of firms to aggressively invest in new assets, expecting higher returns in the long run. Investors may overpay for such firms due to optimism, which results in lower subsequent returns \cite{titman_2004}.


I am going to run also the sentiment index on this too, so 8 factors

" Linear
models are preferred by practitioners because they gen-
erally present readily understandable and interpretable
analysis. In contrast, machine learning approaches,
although useful in uncovering the nonlinear behavior
of and interaction relationships among variables, are
often articulated in a way that makes their output unin-
tuitive, and hence unattractive, to many investment pro-
fessionals. "

Also, you don't even need to do the portfolio sortings like in thesis

Instead of just looking at the average effect of factors, analyzing different percentiles helps us understand how factors behave across low, median, and high return environments.

We are going to do this with the portfolios constructed before, so no need for percentiles


Random Forest Importance compares how important each feature is comparing to others in the same set, it is very useful to offer portfolio guidance such as in  return based style analyis (RBSA). he RF model, however,
possesses an advantage over a standard RBSA in gener-
ally providing a much better fit,

"Because the RF model captures hierarchical
(non-geometric) relationships between factors, it cannot
be understood as a direct analog of an OLS regression
or PCA because it does not convey the individual direc-
tional relationships between factors and assets."

We have to attempt to "beta-tize" the random forest. Beta is basically the elasticity of one variable to another. What we can do is to divide the predicted target variable return by each pre-
dictor return to gain a raw elasticity value for each factor.

\subsection{Random Forest}
- Factor models are generally linear. Which kinds of regression are they usally used with (fama macbeth)? 

Time series regression on six asset pricing models.





\textbf{Sector rotation strategy}

Apply RF variant to build a sector rotation strategy using ARL (Association rule learning). We can use this to deduct inference rule from epirical data

- RL is a rolling-window learning approach, meaning it continuously updates its understanding of market conditions.
Instead of using a fixed model trained once on historical data, it adapts dynamically by re-estimating relationships every 18 months.
This helps capture changing market dynamics—for example, a factor that worked last year may not work the same way today

Two signals are used:
- Random forest predicted excess within a portfolio
- Ratio of shorter-term to longer-term realized volatility (24-month vs. 36-month)

iterative add factor de xem la feature importance co doi ko 