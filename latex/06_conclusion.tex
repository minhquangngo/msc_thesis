
In conclusion, our findings show that augmenting the traditional C4F and FF5 frameworks with liquidity and investor sentiment information yields targeted but meaningful out-of-sample gains while preserving the transparency that makes non-parametric models attractive to practitioners. Shifting from OLS to RF already multiplies explanatory power around six fold and cuts hold-out errors by roughly two orders of magnitude, validating the use of tree ensembles for sector level return forecasting. Layering in the liquidity proxies and the sentiment index produces a second, more nuanced improvement. Forecast accuracy rises significantly (at the 95 \% confidence level or better) in those sectors where illiquidity shocks or behavioural swings matter most (Energy, Materials, Industrials, Utilities and Real Estate in the C4F setting, and all but two sectors in the FF5 setting). Crucially, the models' interpretability is not sacrificed, the pseudo-betas retain the familiar economic meaning of linear betas, allowing finance professionals to trace why predictions move when either traditional or alternative factors change. In sum, enhancing multifactor asset-pricing models with carefully chosen liquidity and sentiment signals delivers sector-specific predictive benefits on top of the sizeable RF baseline gains, all while keeping the rule-based intuition and pseudo-beta diagnostics that render the non-parametric approach practically usable.

We translate each model's predicted excess returns into a straightforward daily rule sets that generates signals for the trading models to automatically assign weights between sectors. Using this approach, signals from liquidity risk and sentiment enhanced Random Forest and OLS models outperform the benchmark S\&P 500 in the volatile 2018 period. Seven of eight unconstrained strategies beat the index and four generated positive absolute returns, with the RF-C4F-E as the best sector rotation strategy model, which shows +5.19\% alpha. However, the Sharpe ratio cannot be statistically proven to be larger than 0, meaning that the returns on investments does not outweigh the risk taken. Therefore, we cannot confidently conclude that the models produce trading signals that will statistically be outperforming the S\&P500. Nevertheless, the results are very promising. Both the models' gains compared to the benchmark index are present in both constrained and unconstrained strategies. 

For asset managers, this is the proposed pipeline: automated data ingestion $\rightarrow$ feature engineering $\rightarrow$ RF forecasting $\rightarrow$ rule extraction. This pipeline delivers a daily sector-weight vector whose rationale can be traced factor-by-factor. When live trades deviate from expectations, global and local attribution via feature importance and pseudo beta diagnostics immediately isolates whether the traditional FF factors ($SMB$, $HML$, $MOM$, $RMW$, $CMA$), the liquidity variables, or the sentiment index drove the surprise deviation. Because liquidity risk and sentiment already proxy two of the most pervasive macro drivers, practitioners can next experiment with complementary state variables (e.g.\ term-structure slope, credit spreads) without inflating model complexity. Moreover, marketing managers can leverage the same factor-attribution framework and daily sector-weight signals to synchronise campaign roll-outs with forecasted shifts in liquidity and sentiment. They can trigger automated alerts for press releases, email blasts or thematic product launches exactly when market conditions turn favourable. The interpretable rule-extraction outputs enable them to calibrate marketing spend and messaging against quantifiable thresholds, ensuring outreach efforts align with the market's evolving risk and sentiment appetite and thereby maximising client engagement and campaign effectiveness.

