Asset pricing models have historically been pivotal for understanding expected returns and constructing investment strategies. Despite the evolution from the Capital Asset Pricing Model (CAPM) of \citeA{sharpe_1964} to the advanced Fama-French Five Factor (FF5) model by \citeA{ff5_2015}, empirical research continues to highlight persistent gaps and unexplained anomalies in asset pricing. Recent literature suggests that integrating additional factors, such as liquidity risk and investor sentiment, could improve the predictive capabilities of these models. Additionally, machine learning techniques, particularly Random Forest (RF), have emerged as promising tools for capturing complex, nonlinear relationships among asset pricing factors. These enhanced predictive capabilities can significantly contribute to the effectiveness of sector rotation strategie. This investment approaches involves reallocating assets between sectors based on their anticipated performance across different economic cycles. Such strategies aim to capitalize on sectors with a forecast for relative outperformance by aligning portfolio allocations with changing economic conditions and market dynamics-such as liquidity risk and investors' sentiment. Consequently, effective sector rotation strategies necessitate accurate predictions informed by sophisticated asset pricing models and advanced analytical methods. This thesis aims to investigate the following research question: 

\begin{center}
    \textit{Can the incorporation of liquidity risk and investors' sentiment into asset pricing rule based models yield a higher performing sector rotation strategy?}
\end{center}

This research is particularly relevant for portfolio managers, financial analysts, and institutional investors seeking improved asset allocation strategies and enhanced returns through sector rotation. Traditional asset pricing models consistently demonstrate limitations in capturing complex, nonlinear market dynamics, particularly concerning liquidity risk and investor sentiment. By integrating liquidity risk and investor sentiment into these established models via advanced machine learning techniques like Random Forest, the research offers practical insights that could enhance predictive accuracy and investment decision-making. Additionally, this thesis is relevant for scholars interested in extending asset pricing literature through innovative machine learning applications, addressing longstanding theoretical and empirical limitations associated with linear factor models and identifying effective mechanisms for systematically extracting actionable trading strategies.