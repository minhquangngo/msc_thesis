Asset-pricing models are indispensable for understanding expected returns and for guiding portfolio construction. Yet, from the Capital Asset Pricing Model (CAPM) of \citeA{sharpe_1964} through the Fama-French Five-Factor (FF5) specification of \citeA{ff5_2015}, linear factor methods potentially leave a persistent share of return factors unexplained, and they struggle with any additional factors unspecified in the model. Two features is constantly mentioned throughout the empirical literature but remain under represented in standard models: liquidity risk, which captures the costs of transacting in thin markets \cite{pastor_2003,acharya_2005}, and investor sentiment, which reflects systematic waves of optimism and pessimism from equity investors \cite{baker_wurgler_2006}.

Recent advances in machine learning offer a route to address these gaps. One novel approach from \citeA{simonian_2019} extends traditional factor models with Random Forests (RF), which promises both higher forecasting power and a transparent explanatory framework from predictors to returns. Beyond in-sample fit, a natural question for financial practitioners is whether such nonlinear, factor enhancing models can be turned into actionable trading rules. Association-rule learning (ARL) is well suited to this task, as again empirically tested by \citeA{simonian_2019}. By mining RF outputs for high-confidence “if-then” patterns, ARL can translate complex predictive relationships into parsimonious decision rules that a portfolio manager can execute. This could be as simple as sector-rotation signals that tilt capital toward industries expected to outperform in the coming day and away from sectors that are forecasted to underperform.

\bigskip
\noindent
With this information, the thesis aims to address one overarching research question by answering two sequential questions:

\textit{Can a machine-learning-augmented factor model both deepen our understanding of cross-sectional stock returns and generate an interpretable sector-rotation strategy that outperforms a broad market benchmark?}

\begin{enumerate}
   \item[(a)] To what extent does enhancing classical multifactor asset-pricing models with additional liquidity and investor-sentiment indicators, improve out-of-sample explanatory and predictive power while preserving interpretability of the non parametric models?\\[4pt]
  \item[(b)] How can the signals produced by such enhanced models be distilled into clear decision rules that guide sector rotation that outperforms S\&P500 index?\\[4pt]
\end{enumerate}


From an academic perspective, the study advances the machine learning and asset pricing literature in three ways. First, it is commonly understood that non-parametric models, while delivering high predictive power, cannot provide the same beta coefficients that finance practitioners require. Prior approaches to reconciling machine-learning accuracy with financial interpretability often relied on ad-hoc surrogate models or novel methods such as SHAP, which lack clear economic meaning. However, the rule set distilled from the Random Forest is interpreted with pseudo-betas and explicit if-then conditions that restore the risk-loading interpretability demanded by empirical-asset-pricing diagnostics and regulatory oversight, while also delivering higher out-of-sample predictions than traditional Fama-French. Second, it extends the Random Forest rule-extraction framework of \citeA{simonian_2019} by embedding liquidity risk per \citeA{pastor_2003} and an enhanced investor sentiment measure from \citeA{ung_2023} alongside the traditional Fama-French factors. By enabling a nonlinear analysis of these macroeconomic variables—rather than treating them only in linear regressions—the thesis overcomes the limited scope of prior studies and captures richer interactions among risk drivers. Third, unlike prior studies that focus primarily on in sample fit and static factor portfolios, this thesis presents rigorous out of sample evidence that interpretable machine learning factors can drive economically significant sector rotation strategies, thereby removing data-mining concerns and bolstering the credibility of the expanding “factor zoo”. Collectively, these contributions bridge the methodological gap between high-accuracy yet opaque machine-learning approaches and economically motivated factor models, demonstrate a practical path for incorporating rich macroeconomic variables into risk-return frameworks without sacrificing transparency or theoretical coherence, and underpin actionable sector-rotation strategies that alleviate data-mining concerns.


From a managerial perspective the decision rules extracted from RF forecasts transform an otherwise opaque algorithm into a set of transparent if then rules. They could then be integrated into dashboards and automated alerts to help portfolio managers and traders time sector rotation trades, scale positions ahead of liquidity squeezes and anticipate sentiment driven volatility spikes. This could also enable risk officers to adjust intraday execution algorithms and stress test funding liquidity scenarios, guiding corporate treasurers and investor relations teams in scheduling equity or debt issuance windows. The insights generated could also provide a real time marketing clock that enables analytics teams to synchronize strategic outreach with the market's risk and sentiment appetite. When the model projects a profitable next day excess return, campaign managers at financial institutions can advance launches of growth oriented offerings such as high beta exchange traded funds, premium margin lending tiers, or thematic equity portfolios based on different sectors. For more traditional marketing analyst at these institutions, they could use the findings to schedule press releases and digital ads to coincide with heightened investor optimism. An example of this strategy in action is AllianceBernstein's launch of its clean energy thematic portfolio. According to AllianceBernstein's disclosures, the institution timed the public release to coincide with a period in which their own (more complex) model registered elevated the right market signals for the clean energy sector \cite{alliance_2024}. In practice the marketing analytics team initiated preparatory measures several months in advance by segmenting target client cohorts, designing phased email campaigns and reallocating digital-advertising budgets toward sustainability-focused creatives. As the model's excess-return forecast for clean energy crossed a certain threshold, the team accelerated client acquisition efforts and issued coordinated press releases, thereby ensuring that the market had already been primed when the real-time model signalled optimal conditions. Conversely, negative forecasted signal could prompt the platform to pivot messaging toward capital-preservation products, loyalty-focused retention workflows. The execution of this campaign is a prime example of how a rules-based framework like the one presented in this paper could be used to maximize the impact of marketing launch by aligning each tactical element with the evolving market risk and sentiment appetite. The insights gathered from this thesis could even assist marketing analytics teams in aligning product launches and client acquisition campaigns with periods of heightened risk appetite to improve conversion efficiency. By converting advanced data science techniques into an interpretable rules based framework this thesis delivers an actionable tool that scales from the trading floor to the C suite.



