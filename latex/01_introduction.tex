Asset pricing models have historically been pivotal for understanding expected returns and constructing investment strategies. Despite the evolution from the Capital Asset Pricing Model (CAPM) of \citeA{sharpe_1964} to the advanced Fama-French Five Factor (FF5) model by \citeA{ff5_2015}, empirical research continues to highlight persistent gaps and unexplained anomalies in asset pricing. Recent literature suggests that integrating additional factors, such as liquidity risk and investor sentiment, could improve the predictive capabilities of these models. Additionally, machine learning techniques, particularly Random Forest (RF), have emerged as promising tools for capturing complex, nonlinear relationships among asset pricing factors. This thesis aims to investigate the following research question: 
\begin{center}
    \textit{Does the incorporation of liquidity risk and sentiment into asset pricing models via Random Forest and rule-based regression yield a higher performing sector rotation strategy?}
\end{center}
