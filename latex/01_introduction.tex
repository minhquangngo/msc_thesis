Asset-pricing models are indispensable for understanding expected returns and for guiding portfolio construction. Yet, from the Capital Asset Pricing Model (CAPM) of \citeA{sharpe_1964} through the Fama-French Five-Factor (FF5) specification of \citeA{ff5_2015}, linear factor methods potentially leave a persistent share of return factors unexplained, and they struggle with any additional factors unspecified in the model. Two features is constantly mentioned throughout the empirical literature but remain under represented in standard models: liquidity risk, which captures the costs of transacting in thin markets \cite{pastor_2003,acharya_2005}, and investor sentiment, which reflects systematic waves of optimism and pessimism from equity investors \cite{baker_wurgler_2006}.

Recent advances in machine learning offer a route to address these gaps. One novel approach from \citeA{simonian_2019} extends traditional factor models with Random Forests (RF), which promises both higher forecasting power and a transparent explanatory framework from predictors to returns. Beyond in-sample fit, a natural question for financial practitioners is whether such nonlinear, factor enhancing models can be turned into actionable trading rules. Association-rule learning (ARL) is well suited to this task, as again empirically tested by \citeA{simonian_2019}. By mining RF outputs for high-confidence “if-then” patterns, ARL can translate complex predictive relationships into parsimonious decision rules that a portfolio manager can execute. This could be as simple as sector-rotation signals that tilt capital toward industries expected to outperform in the coming day and away from sectors that are forecasted to underperform.

\bigskip
\noindent
With this information, the thesis pursues two research questions:

\begin{enumerate}
\item \textbf{Research question 1}\
\textit{Does augmenting established linear asset-pricing models with liquidity-risk and sentiment factors, estimated via Random Forests, yield significantly higher out-of-sample explanatory power while retaining interpretability?}
\item \textbf{Research question 2}\
\textit{Given such an RF-based factor model, can its forecasts be combined with association-rule learning to derive simple, transparent sector-rotation rules that outperform the S\&P 500 index?}
\end{enumerate}

Addressing these questions contributes to both academic and applied finance. Academically, the work tests whether nonlinear models along with liquidity risk and sentiment factors yield higher explanatory power than linear models. From a practitioner's standpoint, we demonstrate how these 'black-box' models can be translated into transparent, rules-based framework that is not only implementable and interpretable, but also delivers meaningful insights for portfolio managers, financial analysts, and institutional investors. 