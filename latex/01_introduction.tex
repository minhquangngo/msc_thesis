
Asset-pricing models form the analytical backbone of modern finance: they turn noisy market data into estimates of expected returns and, by doing so, provide the foundation for portfolio construction, risk management, and capital-allocation decisions. From the Capital Asset Pricing Model (CAPM) of \citeA{sharpe_1964} to the Fama-French Five-Factor (FF5) specification of \citeA{ff5_2015}, the standard practice has been to express expected returns as linear combinations of a small set of “risk factors”. Despite their simplicity and enduring popularity, these linear models have been empirically shown to leave portions of the cross-section of returns unexplained and offer limited guidance when new sources of risk emerge. Persistent mispricing anomalies that fall outside the cover of the traditional Fama French factors testify to these gaps and motivate the search for more robust asset pricing frameworks.

% Two economically intuitive dimensions of risk receive recurrent attention in the empirical literature yet remain under-represented in mainstream factor models. Liquidity risk captures the price impact and transaction cost penalties associated with trading in thin or stressed markets \cite{pastor_2003,acharya_2005}. Investor sentiment reflects systematic waves of optimism and pessimism that can temporarily decouple prices from the intrinsic value of companies \cite{wurgler_2007}.  A growing body of evidence shows that both dimensions command return premia and help explain well-known anomalies, suggesting that omitting them leaves valuable information on the table for academics and practitioners alike. Recent advances in machine learning point to a practical route for closing these gaps without sacrificing interpretability. In particular, we adopt the pseudo-beta framework of \citeA{simonian_2019}, which converts the variable importance of a RF ensemble into loadings that are conceptually analogous to OLS betas, thereby preserving the economic intuition of factor models while accommodating the non-linearities of liquidity and sentiment effects.

% Building on \citeA{simonian_2019}, this thesis investigates whether augmenting classical factor models with a superior specification such as RF can translate superior in-sample fit into actionable out of sample performance. Although Random Forests excel at capturing the non linear interactions between the so called `factor zoo', the sheer volume of their raw ensemble predictions offers little practical guidance for day to day portfolio allocation.  Association Rule Learning (ARL) closes this implementation gap by distilling those forecasts into a handful of transparent if-then rules, yielding deterministic signals that portfolio managers can audit, back-test, and communicate with ease. Equity sectors provide a natural arena for putting such rules to work.  Firms within a sector share common cash-flow drivers, regulatory settings, and macroeconomic sensitivities, so sector-level return dispersion is often pronounced even when aggregate market moves are calm. Because most institutional mandates already permit tactical deviations from a strategic sector benchmark, a sector rotation strategy based on ARL filtered Random-Forest signals integrates smoothly with existing governance frameworks and remains scalable at institutional capacity. 

These inconsistencies of the market remind both academics and practitioners that even the most preferred multifactor frameworks still omit economically important factors. Two such omissions stand out. One such omission is liquidity risk, which reflects the extra compensation investors demand for trading in illiquid or stressed market conditions. This is an aspect that standard multifactor models only partially capture \cite{pastor_2003,acharya_2005}. Investor sentiment in the form of optimism and pessimism could push prices away from its intrinsic value, affecting subsequent returns of equities \cite{wurgler_2007}. Ignoring these dimensions not only hampers our scientific understanding of asset pricing, but it also leaves exploitable information on the table for real-world portfolio construction.

Yet simply tacking new variables onto an ever-expanding set of factors is not an easy task. What is needed is a modeling framework that (i) accommodates the nonlinear interactions that liquidity and sentiment naturally exhibit, and (ii) preserves the economic intuition that has made linear factor models so influential. Recent machine-learning advances suggest a viable path. Ensemble methods such as Random Forests (RF) can flexibly learn complex return drivers, while post-hoc interpretation techniques such the pseudo-beta of \citeA{simonian_2019} can translate their output back into the language of betas. If those insights can be distilled into a handful of transparent rules, they could generate a sector-rotation strategy that fits seamlessly within institutional mandates. Accordingly, this thesis asks:

\textit{Can a machine-learning-augmented factor model both deepen our understanding of cross-sectional stock returns and generate an interpretable sector-rotation strategy?}

which is answered via two sub-questions:
\begin{enumerate}
   \item[(a)] To what extent does enhancing classical multifactor asset-pricing models with additional liquidity and investor-sentiment indicators, improve predictive power and fit while preserving interpretability of the non-parametric models?\\
  \item[(b)] Can the signals produced by such enhanced models be distilled into clear decision rules that guide sector rotation that outperforms S\&P500 index?\\
\end{enumerate}

To answer these questions, we (i) extend the C4F and FF5 models with liquidity and sentiment measures and translate Random Forest variable importance scores into pseudo-beta loadings following the framework of \citeA{simonian_2019}, (ii) compare out-of-sample explanatory and predictive performance across classical linear models, non-parametric specifications, and alternative RF implementations, and (iii) apply Association Rule Learning (ARL) to distill stable, interpretable pseudo-beta signals into daily sector rotation rules. By bridging the flexible explanatory power of machine learning with the transparency demanded by investment professionals, this study contributes both to academic debates on factor completeness and to the practical design of rule-based allocation strategies. 


From an academic perspective, the study advances the machine learning and asset pricing literature in three ways. First, it is commonly understood that non-parametric models, while delivering high predictive power, cannot provide the same beta coefficients that finance practitioners require. Prior approaches to reconciling machine-learning accuracy with financial interpretability often relied on ad-hoc surrogate models or novel methods such as SHAP, which lack clear economic meaning. However, in this study, Random Forest is interpreted with pseudo-betas that restore the interpretability demanded by financial practitioners and regulatory oversight, while also delivering higher out-of-sample predictions than traditional Fama-French. Second, it extends the Random Forest rule extraction framework of \citeA{simonian_2019} by embedding liquidity risk per \citeA{pastor_2003} and an enhanced investor sentiment measure from \citeA{ung_2023} alongside the traditional Fama-French factors. By following a nonlinear analysis of these macroeconomic variables, the thesis overcomes the limited scope of prior studies and captures richer interactions among risk drivers. Third, unlike prior studies that focus primarily on in-sample fit and static factor portfolios, this thesis presents out-of-sample evidence that interpretable machine learning factors can drive economically significant sector rotation strategies, thereby removing data-mining concerns and increasing the credibility of the expanding “factor zoo”. Collectively, these contributions bridge the academic gap between high accuracy yet opaque machine-learning approaches and traditional asset pricing models. They demonstrate a practical path for incorporating rich macroeconomic variables into risk-return frameworks without sacrificing transparency or theoretical coherence, and establish the validity of actionable sector rotation strategies.


From a managerial perspective, the decision rules of the sector rotation strategy extracted from RF forecasts are easily interpreted with a set of if-then conditions. They could then be integrated into dashboards and automated alerts to help portfolio managers and traders time sector rotation trades, scale positions ahead of liquidity squeezes and anticipate sentiment-driven volatility spikes. This could also enable risk officers to adjust intraday execution algorithms and stress test funding liquidity scenarios, guiding corporate treasurers and investor relations teams in scheduling equity or debt issuance windows. The insights generated could also provide a real-time marketing clock that enables analytics teams to synchronize strategic outreach with the market's risk and sentiment appetite. When the model projects a profitable next day excess return, campaign managers at financial institutions can push launches of the institutions' offerings such as high beta exchange traded funds, premium margin lending tiers, or thematic equity portfolios based on different sectors. For more traditional marketing analysts at these institutions, they could use the findings to schedule press releases and digital ads to coincide with heightened investor optimism. An example of this strategy in action is AllianceBernstein's launch of its clean energy thematic portfolio. According to AllianceBernstein's disclosures, the institution timed the public release to coincide with a period in which their own (more complex) model registered elevated the right market signals for the clean energy sector \cite{alliance_2024}. In practice the marketing analytics team initiate preparatory measures several months in advance by segmenting target client cohorts, designing phased email campaigns and reallocating digital-advertising budgets toward sustainability-focused creatives. As the model's excess-return forecast for clean energy crossed a certain threshold, the team accelerated client acquisition efforts and issued coordinated press releases, thereby ensuring that the market had already been primed when the real-time model signalled optimal conditions. Conversely, negative forecasted signal could prompt the platform to pivot messaging toward capital-preservation products, loyalty-focused retention workflows. The execution of this campaign is a prime example of how a rules-based framework like the one presented in this paper could be used to maximize the impact of marketing launch by aligning each tactical element with the evolving market risk and sentiment appetite. By converting advanced data science techniques into an interpretable rules-based framework this thesis delivers an actionable tool that scales from the trading floor to the C-suite.

The remainder of the study is organised as follows.  Section~\ref{sec:litrev} surveys the current literature on multifactor asset pricing models, machine-learning applications in finance, and the roles of liquidity risk and investor sentiment.  Section~\ref{sec:method} details the methodological framework, describing factor construction, the RF forecasting setup, and the ARL procedure that converts model outputs into sector rotation signals.  Section~\ref{sec:data} documents the data sources, sample selection and feature-engineering steps. Section~\ref{sec:results} presents the empirical results: model-interpretability diagnostics, forecasting comparisons, Diebold-Mariano tests, and the performance of both unconstrained and constrained rotation strategies. Section~\ref{sec:discussion} discusses the findings in light of the research questions, highlighting limitations, outlining managerial implications, and suggesting directions for future research. Finally, Section~\ref{sec:conclusion} concludes the thesis by summarizing the main contributions and implications of the study. 
