This literature review will examine the theoretical evolution of standard asset pricing models, beginning with the single-factor Capital Asset Pricing Model (CAPM) and extending to the multi-factor Fama-French frameworks. It will highlight the limitations of traditional statistical approaches in asset pricing and explore how advancements in machine learning methodologies may address these shortcomings. Then, the existing literature of the proposed enhancements to the asset pricing models will be reviewed. Specifically, the focus will be on the justification for incorporating liquidity risk and investor sentiment as explanatory factors. The ways in which contemporary researchers integrate these factors into their models will also be discussed. Furthermore, the application of asset pricing models within a sector rotation strategy will be explored, including an assessment of how this approach has been implemented in recent studies.


\subsection{A review of CAPM and its development}

The single-factor CAPM introduced by \citeA{sharpe_1964} links asset returns to market beta, measuring systematic market risk. Although theoretically appealing due to its simplicity, empirical testing has repeatedly shown deviations, such as unexplained outperformance of small-cap and value stocks (\cite{capm_2004}). In response, \citeA{ff3_1993} expanded CAPM to include size and value factors (FF3), significantly improving explanatory power. Nonetheless, further anomalies remained unexplained, including investment and profitability effects identified by \citeA{titman_2004} and \citeA{novymarx_2013}, as well as momentum effects observed by \citeA{cahart_1997}. To address this, \citeA{ff5_2015} introduced profitability and investment factors into the FF5 model, improving upon previous models but still leaving notable gaps unexplained, such as the momentum anomaly and returns on specific small-cap stocks (\cite{sarwarff5}, \citeA{benammar_2018}, \citeA{cakici_2015}). In sum, over decades the asset pricing model development (CAPM $\rightarrow$ FF3 $\rightarrow$ C4F $\rightarrow$ FF5) was driven by the need to address empirical gaps. Each new factor was added to account for a systematic return pattern unexplained by prior models. While these multifactor models capture much of the cross-section, research continues to find gaps, suggesting even more factors may be needed.


\subsection{Role of machine learning in factor models}

Traditional asset pricing models, including CAPM and Fama-French variants, are primarily linear, limiting their capacity to capture nonlinear market dynamics. Early recognition by \citeA{mcdonald_1962, mcdonald_1983} highlighted the need for polynomial or nonlinear regression functions in financial analysis. Machine learning techniques have shown promise in addressing these limitations by modeling complex, nonlinear interactions among predictors. For instance, \citeA{gu_2020} and \citeA{freyberger_2018} demonstrate superior predictive capabilities of ML models, including Random Forests (RF) and neural networks, over traditional linear methods. However, such improvements come at the cost of interpretability, known as the "black-box" problem \cite{brozek_2024}. Researchers have therefore pursued model-specific methods, including surrogate models or decision trees, to balance predictive accuracy and transparency \cite{simonian_2019}. Specifically, \citeA{simonian_2019} applied Random Forests to enhance the Carhart four-factor model (C4F), showing increased predictive power while maintaining interpretability through Rule-based strategies and feature importance metrics.

\subsection{Emerging additions to factor models}
An attractive feature of ML-enhanced models is their ability to incorporate a wide array of features that traditional econometrics models cannot - at least without constricting assumptions. This allows for the opportunity to include more factors within the factor models. Two factors that are frequently highlighted in the literature as valuable additions to factor models are liquidity risk and investor sentiment. The following discussion examines why these two factors are considered for asset pricing models and the findings when they are incorporated.

%%Maybe problems with factor zoo? 

\subsubsection{Liquidity Risk}
%%%%%Why liquidity is considered for asset pricing
A liquid investment is characterized by the ability to be quickly bought and sold without significantly impacting its price. Liquidity risk emerges when assets experience a lack of market participants, leading to increased price volatility and higher transaction costs. Empirical evidence from \citeA{pastor_2003} supports the claim that liquidity risk is a priced factor, as stocks sensitive to market-wide liquidity fluctuations command a premium; investors demand additional compensation for the potential difficulty in trading assets during periods of reduced liquidity. Furthermore, when liquidity deteriorates or "dries up," this risk intensifies, amplifying transaction costs and forcing leveraged investors into costly liquidation, exacerbating price impacts. \citeA{amihud_1986} and \citeA{amihud_2002} demonstrate that bid-ask spreads—an established proxy for liquidity—represent the cost of transacting in illiquid assets. Investors holding assets for shorter periods incur these transaction costs frequently, thus demanding higher returns to offset illiquidity. Moreover, expectations of future illiquidity further increase required returns and depress asset prices, providing strong justification for liquidity's integration as a priced systematic risk factor within asset pricing models \cite{amihud_2002}.

The literature offers several approaches for explicitly integrating liquidity into asset pricing frameworks. \citeA{acharya_2005} extended the traditional CAPM by adding liquidity-related risk factors, highlighting that market risk alone is insufficient without accounting for the sensitivity of asset returns and liquidity to broader market conditions. Although this liquidity-adjusted CAPM improved explanatory power over the standard CAPM, it still did not fully explain phenomena like the book-to-market effect. Similarly, \citeA{brennan_1998} examined trading volume as a liquidity proxy within Fama-French frameworks, confirming that liquidity is priced in the market and reduces the significance of the size (SMB) factor, given the inherent illiquidity in smaller firms. However, findings by \citeA{li_2019} were less conclusive; they replicated liquidity risk factors based on historical liquidity betas but observed only marginal improvements to asset pricing models such as FF3 and Carhart's four-factor (C4F) model. Notwithstanding these mixed outcomes, liquidity risk remains empirically significant in various markets, as demonstrated by \citeA{jang_2012}, whose liquidity-enhanced CAPM notably captured return anomalies better than traditional models, particularly during financial crises, underscoring liquidity’s potential for addressing systematic return anomalies.

\subsubsection{Investors' sentiment}
Traditional asset pricing models assume competition among rational investors, leading to market equilibrium where asset prices reflect their discounted future cash flows. Even if individual investors behave irrationally, such "noise behavior" should, in theory, self-correct at an aggregate market level \cite{friedman_1953}. However, behavioral finance literature argues that irrational investors—those who trade without reliance on fundamental or technical analyses—can significantly impact prices through "investor sentiment". Although definitions vary, \citeA{wurgler_2007} provide a broadly accepted definition, describing investor sentiment as "a belief about future cash flows and investment risks that is not justified by the facts at hand." Empirical evidence supports sentiment's influence on asset pricing; for instance, \citeA{smales_2017} used volatility indices as proxies for sentiment and demonstrated causality by showing sentiment shifts preceded significant stock price movements. Similarly, \citeA{wurgler_2007} introduced the concept of a "sentiment seesaw," illustrating that speculative stocks—typically small, young, high-volatility, or non-dividend-paying—tend to become overvalued during periods of high sentiment and undervalued during low sentiment, resulting in predictable subsequent price corrections.
The integration of investor sentiment into asset pricing models has been explored primarily through extensions of the Fama-French three-factor (FF3) and Carhart four-factor (C4F) models. For instance, \citeA{stambaugh_2015} demonstrated that augmenting the FF3 model with sentiment-based idiosyncratic volatility improves explanatory power, capturing sentiment-induced mispricing previously overlooked by traditional factors. Similarly, \citeA{wurgler_2006} highlighted how the SMB factor within the C4F model interacts with sentiment, indicating that smaller stocks are disproportionately influenced by investor sentiment, a pattern not adequately captured by size alone. Additionally, \citeA{antoniou_2016} found that CAPM's central prediction—that high-beta stocks yield higher returns—only holds under pessimistic market sentiment; during optimistic periods, investor sentiment causes high-beta stocks to become overpriced and subsequently underperform. Further, \citeA{fongtoh_2014} analyzed the "MAX" anomaly, whereby stocks with high maximum daily returns ("high MAX") underperform, a phenomenon explained through investor gambling behaviors driven by sentiment. However, despite consensus regarding sentiment's importance, there remains no unified definition or measure, complicating efforts to standardize its implementation in asset pricing models.



